\documentclass{beamer}

\usepackage{fancyvrb}


\title{Functional Reactive Programming in Haskell}
\author{Lee Avital}

\begin{document}


\maketitle

\begin{frame}
  \frametitle{Functional Reactive Programming (FRP)}

  \begin{itemize}
    \item State is a function of time.
    \item State is a function of outside input (key presses, mouse click)
    \item Discrete vs Continuous
  \end{itemize}
\end{frame}

\begin{frame}[fragile]
  \frametitle{Key Abstraction: Event}

    % \begin{semiverbatim}
    % data Direction = Up | Down
    % getDir = (Just Down)
    % getDir  = (Just Up)
    % \begin{semiverbatim}

  \begin{itemize}
    \item A "stream" of values.
    \item Functor
    \item Monoid (mappend, mempty)
    \item Applicative
    \item \texttt{ type E a = [T, a] }
  \end{itemize}

  \begin{Verbatim}
    data Direction = Up | Down
    getDir 'j' = (Just Down)
    getDir 'k' = (Just Up)
    echar :: Event t Char
    getDir <$> echar :: Event t Direction
  \end{Verbatim}

\end{frame}




\begin{frame}[fragile]
  \frametitle{Key Abstraction: Behavior}

  \begin{itemize}
    \item A continuous function from time to a value.
    \item \texttt{type Behaviour a = T $\rightarrow$  a}
  \end{itemize}
  \begin{Verbatim}
    eNum :: Event Int
    (\x -> (+x)) <$> eNum :: Event (Int -> Int)
    accumB 0 eNum :: Behaviour Int -- running sum
    stepper 0  eNum :: Event Int -- last integer
  \end{Verbatim}
\end{frame}


\begin{frame}[fragile]
  \frametitle{What I did}

  \begin{itemize}
    \item Reactive Banana --- FRP implementation
    \item HSCurses --- ncurses binding
    \item Reactive Banana + HSCurses = Cursed Banana
  \end{itemize}

\end{frame}


\end{document}
